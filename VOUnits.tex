%%%%%%%%%%%%%%%%%%%%%%%%%%%%%%%%%%%%%%%%%%%%%%%
%  For an conversion via cgiprint (HTX):
%  See http://vizier.u-strasbg.fr/local/man/cgiprint.htx
\def\ifhtx{\iffalse}    % Lines used only for the HTML version
\ifhtx
% . . .
% . . . Definitions in HTX context
% . . .
\else
\documentclass[11pt,notitlepage,onecolumn]{ivoa}
% . . .
% . . . Definitions in LaTeX context
% . . .
\fi
%%%%%%%%%%%%%%%%%%%%%%%%%%%%%%%%%%%%%%%%%%%%%%%

\def\SVN$#1: #2 ${\expandafter\def\csname SVN#1\endcsname{#2}}
\SVN$Revision$
\SVN$Date$
\SVN$HeadURL$

%% Comment/uncomment lines below to follow your LateX distribution...

\usepackage{natbib} % use author-year citations

\usepackage{prettyref} % ensure consistent cross-references
\newrefformat{sec}{Sect.~\ref{#1}}
\newrefformat{appx}{Appx.~\ref{#1}}
\newrefformat{fig}{Fig.~\ref{#1}}
\newrefformat{tab}{Table~\ref{#1}}
\usepackage{varioref}
\newrefformat{tabx}{Table~\vref{#1}}

\usepackage{supertabular,multicol}

% Physical units in \rm.  Unstarred version includes leading
% \thinspace.  Starred version doesn't, and is used when referring to
% the unit by itself (eg axis is $B/\units*T$), and is not qualifying
% a number
\makeatletter
\def\units{\@ifstar{\let\un@tsspace\relax    \un@ts}%
                   {\let\un@tsspace\thinspace\un@ts}}
\newcommand{\un@ts}[1]{{\let~\thinspace
  \ifmmode
    \un@tsspace\mathrm{#1}%
  \else
    \nobreak$\un@tsspace\mathrm{#1}$%
  \fi}}

\usepackage{verbatim} % for \verbatiminput
\def\verbatim@font{\fontsize{9}{11}\selectfont\ttfamily}
% \DeclareRobustCommand{\^}{%
%    \ifmmode\nfss@text{\textasciicircum}\else\textasciicircum\fi}

\makeatother

% abbreviation for 'e.g.', which (a) gets spacing right after the full
% stop, and (b) allows us to change the punctuation globally if we
% decide to.
\def\eg{e.g.~}

%%
%% If document is processed with latex, dvips and ps2pdf
%%
\ifx\pdftexversion\undefined
  \usepackage[dvips]{graphicx}
  \DeclareGraphicsExtensions{.eps,.ps}
%% Uncomment following line if you want PDF thumbnails
%  \usepackage[ps2pdf]{thumbpdf}
% for old hyperref, use:
  \usepackage[ps2pdf]{hyperref}
%% for recent hyperref, use:
%  \usepackage[ps2pdf,bookmarks=true,bookmarksnumbered=true,hypertexnames=false,breaklinks=true,%
%  colorlinks,linkcolor=blue,urlcolor=blue]{hyperref}

%%
%% else if document is processed with pdflatex
%%
\else
  \usepackage[pdftex]{graphicx} %% graphics for pdftex (supports .pdf .jpg .png)
  \usepackage{epstopdf}         %% requires epstopdf
%% this is to support .ps files :
  \makeatletter
  \g@addto@macro\Gin@extensions{,.ps}
  \@namedef{Gin@rule@.ps}#1{{pdf}{.pdf}{`ps2pdf #1}}
  \makeatother
%% comment above lines if you have included ps files
%\DeclareGraphicsExtensions{.pdf,.jpg,.png}
%% Uncomment following line if you want PDF thumbnails
%  \usepackage[pdftex]{thumbpdf}
%% for old hyperref, use:
%  \usepackage[ps2pdf]{hyperref}
% for recent hyperref, use:
 \usepackage[pdftex,bookmarks=true,bookmarksnumbered=true,hypertexnames=false,breaklinks=true,%
  colorlinks,allcolors=ivoacolor]{hyperref}
  \pdfadjustspacing=1
\fi
\usepackage[final]{pdfpages}
%\usepackage{tabulary} %%
%%  Header of the document...
%%
% Provide a title for your document
\title{Units in the VO}
% Give date and version number
\date{1.0-20130724}

% Choose one document type from below
%\ivoatype{IVOA Note}
%\ivoatype{IVOA Working Draft}
\ivoatype{IVOA Proposed Recommendation}
%\ivoatype{IVOA Recommendation}

\version{1.0}
% Give author list: separate different authors with \\
% You can add email addresses with links \url{mailto:yourname@ivoa.net}
\author{S\'{e}bastien Derri\`ere,
Norman Gray,
Mireille Louys,
Jonathan McDowell,
Fran\c{c}ois Ochsenbein,
Pedro Osuna,
Anita Richards,
Bruno Rino,
Jesus Salgado}
\editor{S\'{e}bastien Derri\`{e}re}

\urlthisversion{\footnotesize{\url{http://www.ivoa.net/Documents/VOUnits/20130724/}}}
\urllastversion{\footnotesize{\url{http://www.ivoa.net/Documents/VOUnits/}}}
\previousversion{\footnotesize{\url{http://www.ivoa.net/Documents/VOUnits/20130429/}}}
%\previousversion{\footnotesize{\url{http://www.ivoa.net/Documents/VOUnits/20130225/}}}
%\previousversion{\footnotesize{\url{http://www.ivoa.net/internal/IVOA/UnitsDesc/WD-VOUnits-v1.0-20120522.pdf}}}
%\urlthisversion{\footnotesize{\url{http://www.ivoa.net/Documents/VOUnits/20130225/}}}
%\previousversion{\footnotesize{\url{http://www.ivoa.net/internal/IVOA/UnitsDesc/WD-VOUnits-v1.0-20120718.pdf}}}
%\previousversion{\footnotesize{\url{http://www.ivoa.net/Documents/VOUnits/20120801/}}}



%%%%%%%%%%%%%%%%%
%mir \documentclass[12pt]{article}
%\usepackage{graphicx}
%\usepackage{hyperref}
%\usepackage{psfig}
%\usepackage{html}
%\usepackage{epsf}
%\usepackage{lscape}
%mir \textheight 9.0in \hoffset -0.5in \voffset -0.5in
%\newcommand{\Sensitiv}{Variation}
\newcommand{\Sensitiv}{Sensitivity}
\definecolor{orange}{rgb}{1.0,0.5,0.0}
\newcommand{\unit}[1]{\textbf{\textsf{\color{orange}#1}}}
\usepackage[T1]{fontenc}
\usepackage{longtable}
\usepackage{multirow}
%\font\symbo=psyr at 10pt
%\def\micro{{\symbo \char109}}
\def\micro{{\ensuremath \mu}}

%Mir colors definitions
\newcommand{\bleu}[1]{\textcolor[rgb]{0.00,0.00,1.00}{#1}}
\newcommand{\blue}{\textcolor{blue}}
\newcommand{\violet}{\textcolor[rgb]{0.50,0.00,0.50}}
\newcommand{\brown}{\textcolor[rgb]{0.50,0.10,0.10}}
%%%%%%%%%%%%%%%%%

%\usepackage{showlabels}



\begin{document}
\maketitle % print header in standard form
\thispagestyle{empty}
\begingroup
%%% These keywords are substituted by Subversion
% (yes, I know that I could just enable keyword substitution on the
% main file, but having used a non-subversion DVCS for a while, the
% idea of having the VCS system edit files makes me itchy!).
\def\SVN$#1: #2 ${\expandafter\def\csname SVN#1\endcsname{#2}}
\SVN$Revision$ % or any SVN keyword
\SVN$Date$
\SVN$HeadURL$
\SVN$Header$

\vfill
%%\hbox to \textwidth{\hfil\tiny Volute: \SVNRevision, \SVNDate}
\hbox to \textwidth{\hfil\tiny code.google.com/p/volute, rev\SVNRevision, \SVNDate}
%\hbox to \textwidth{\hfil\tiny Volute: \SVNHeader}
\endgroup
\newpage
\tableofcontents 
\newpage
\listoftables
\newpage
\section*{Abstract}

This document describes common practices in manipulating
units in astronomical metadata and defines a means of consistent
representation within VO services.

The core of the document contains the recommended rules for writing string representations 
of unit labels, called VOUnits.

\section*{Status of this document}

This is an IVOA Proposed Recommendation made available for public review.
It is appropriate to reference this document only as a recommended standard 
that is under review and which may be changed before it is accepted as a full recommendation.

%This is an IVOA Working Draft for review by IVOA members and
%other interested parties. It is a draft document and may be updated,
%replaced, or rendered obsolete by other documents at any time. It is 
%inappropriate to use IVOA Working Drafts as reference materials or to cite 
%them as other than ``work in progress''.

This document is a substantial update of the previous version 0.2 that
was written within the Data Model IVOA Working Group. As decided in previous
IVOA interoperability meetings, the Semantics working group is now in charge 
of the document. This document is intended to become a full IVOA recommendation,
following agreement within the community and standard IVOA recommendation process.

The place for discussions related to this document is the
Semantics IVOA mailing list {\tt semantics\@@ivoa.net}.

%\def\UrlFont{\small}

A list of current IVOA recommendations and other technical documents can be found at
\url{http://www.ivoa.net/Documents/}.

\section*{Acknowledgements}

We thank all those participants in IVOA and EuroVO workshops who have
contributed by exposing use cases and providing comments, especially
Markus Demleitner,
Paddy Leahy,
Jeff Lusted,
Arnold Rots,
Mark Taylor,
Brian Thomas
and recent contributors on the DM forum.
%This document is just an update of the previous draft discussed in
%Strasbourg Interop meeting in 2009. It will be completed with
%appropriate grammar and tools to check different levels of consistency
%for a unit string used in the VO context.


\section{Introduction}

This document describes a standardised use of units in the VO
(hereafter simply `VOUnits').  It aims to describe a syntax for unit
strings which is as far as possible in the intersection of existing
syntaxes, and to list a set of `known units' which is
the union of the `known units' of those standards.
We \emph{recommend}, therefore, that applications which write out
units should do so using \emph{only} the VOUnits syntax, and that
applications reading units should be able to read \emph{at least} the
VOUnits syntax, plus all of the units of \prettyref{sec:knownunits}.

We also provide, for information, a set of self- and mutually-consistent
machine-readable grammars for all of the syntaxes discussed.

The introduction gives the motivation for
this proposal in the context of the VO architecture, from the legacy 
metadata available in the resource layer, to the requirements of the various 
VO protocols and standards and applications.

This document is organised as follows: \prettyref{sec:proposal}
details the proposal for VOUnits. \prettyref{sec:useCase} lists some
use cases and reference implementations.  In \prettyref{appx:current},
there is a brief review of current practices in the description and
usage of units; in \prettyref{appx:comparisons} there is a detailed
discussion of the differences between the various syntaxes; and
in \prettyref{appx:grammar} there are formal (yacc-style) grammars of
the four syntaxes discussed.

The normative content of this document is \prettyref{sec:proposal} and \prettyref{appx:vougrammar}.

\subsection{Units in the VO Architecture}

\begin{figure}[htbp]
  \centerline{\includegraphics[width=0.9\textwidth]{unitsInIVOA.pdf}}
  \caption{Units is a core building block in the VO.  Most parts of the
  architecture rely on it: the User Layer with tools and clients, the
  Resource Layer with data.  Protocols, registries entries, and
  data models also re-use these Units definitions.}
  \label{fig:architecture}
\end{figure}

Generally, every quantity provided in astronomy has a unit attached to
its value or is unitless (\eg a ratio, or a numerical multiplier).  

The Units lie at the core of the VO architecture, as can be seen in \prettyref{fig:architecture}.
Most of the existing data and metadata collections accessible in the resource
layer do have some legacy units, which are mandatory for any scientific use of
the corresponding data.  Units can be embedded in data (\eg FITS headers) or be
implied by convention and/or (preferably) specified in metadata.


Units also appear in the VOTable format \citep{ochsenbein11}, through the use
of a {\tt unit} attribute that can be used in the {\tt FIELD}, {\tt PARAM} and {\tt INFO} 
elements. Because of the widespread dependency of many other VO standards on VOTable,
these standards inherit a dependency on Units.

The Units also appear in many Data Models, through the use of dedicated elements in
the models and schemas.
At present, each VO standard either refers to some external reference document, or 
provides explicit examples of the Units to be used in its scope, on a case-by-case
basis.

The registry records can also contain units, for the description of table metadata.
The definition of VO Data Access protocols uses units by specifying in which units the input
parameters have to be expressed, or by restricting the possible units in which some 
output must be returned.

And last but not least, tools can interpret units, for example to display
heterogeneous data in a single diagram by applying conversions to a reference 
unit on each axis.

\subsection{Adopted terms and notations\label{sec:notations}}

Discussions about units often suffer from misunderstandings arising from cultural
differences or ambiguities in the adopted vocabulary. For the sake of clarity, in this 
document, the following concepts are used:

\begin{itemize}
\item a \textbf{quantity} is the combination of a (numerical) {\em value}, measured for a {\em concept} and expressed in a given {\em unit}. 
In the VO context, the nature of the concept can be expressed with a UCD or a utype. This document does not address the full issue of
representing quantities, but focusses on the {\em unit} part.
\item a \textbf{unit} can be expressed in various forms: in natural language (\eg \emph{metres per second squared}), with a combination
of symbols with typographic conventions (\eg m s$^{-2}$), or by a simplified text label (\eg \unit{m.s-2}). VOUnit deals with the
label form, which is easier to standardize, parse and exchange. A VOUnit corresponds in the most general case to a combination of
several (possibly prefixed) symbols with mathematical operations expressed in a controlled syntax.
\item a \textbf{base unit} is represented by a \textbf{base symbol}, with unambiguous meaning.
\item a \textbf{prefix} or \textbf{scale factor} is prepended to a
\textbf{base symbol} to scale it, most typically by powers of ten.
\item a \textbf{symbol} or \unit{sym} is either a base symbol or a prefixed base symbol.
\end{itemize}

%% Remark: some complex questions, more related to data modeling than to units, such as how a quantity 
%% is associated to its measurement error, or how groups of coordinates are described, are not addressed in this
%% document. They can always be broken down, with appropriate modeling, into smaller bits to which VOUnits can
%% be applied.


\subsection{Purpose of this document}
\label{sec:purpose}

The purpose of this document is to provide a reference specification of how
to write VOUnits, in order to maximize interoperability within the VO;
the intention is that VOUnit strings should be reliably
parseable by computer, with a single interpretation.
This is broadly the case for the other existing
unit-string syntaxes, although there are some slight ambiguities in
the specifications of these syntaxes (cf \prettyref{appx:grammar}).
We therefore include a set of self- and mutually-consistent
machine-readable grammars for all of the syntaxes discussed.

VOUnits will not try to reinvent the wheel, and will be as compliant as possible with
legacy metadata in major archives, and astronomers' habits.

%% By explicitly stating what must (and must not) be used, what may be used, and
%% giving clear recommendations as to the preferred practices, it should avoid some
%% common misinterpretations and confusions that are often encountered when dealing with
%% units.
In particular:
\begin{itemize}
\item We describe (\prettyref{appx:current}) a number of existing unit
  syntaxes, and mention some ambiguities in their
  definition. Application authors should expect to encounter each of
  the syntaxes mentioned in this document (FITS, OGIP and CDS); all of
  these are broadly endorsed by this specification.
\item In addition to the unit syntaxes described above, there are
  multiple specifications of base and known units
  (we refer, in particular, to 
  specifications from BIPM, ISO/IEC and the IAU);
  %\citet{si-brochure,std:iec80000-13,iau12});
  these are broadly, but not completely, mutually consistent.
\item Where there are some ambiguities in, or contradictions between,
  these various specifications, we recommend that application authors should
  resolve them as indicated in this specification.  
\item This document defines a syntax, called `VOUnits', which is as
  far as is feasible in the intersection of the three existing syntaxes, and
  which we recommend that applications should use when writing unit
  strings.  This aim is not quite
  possible in fact, and the mild deviations from it are discussed
  below in \prettyref{sec:proposal} and \prettyref{appx:grammar};
  there is a summary of the various units in \prettyref{tabx:knownunits}.
\end{itemize}

% Data providers are encouraged to follow the VOUnits specifications for expressing
% their metadata. And application developers can rely on these specifications in order
% to know what VOUnits they should expect to face.



\subsection {What this document will not do}
\label{sec:outofscope}

This Recommendation does \textbf{not} prescribe what units data providers
employ, nor demand that a given quantity be expressed in a
unique way (\eg all distances in \unit{m}).  So long as data is labelled 
in a recognised system, a
translation layer can be provided. Data providers can customise the
translation tools if required. Depending on preference and the
operations required, the user may have a choice of units for his or her
query and for the result.

This Recommendation does not discuss \emph{quantities} at all.  That
is, we do not discuss the combination of number and unit which refers
to a particular physical measurement, such as `2$\mathrm m\,\mathrm
s^{-1}$'.  Though this might appear to be a trivial extension, it
raises questions of the representation of decimal numbers, the
representation of uncertainties, questions of unit conversion, and
other data-modelling imponderables which have in the past, possibly
surprisingly, generated a great deal of rhetorical heat within the
IVOA without, so far, a generally acceptable resolution.

%% VOUnits do not specify how transformation of quantities is done. But VOUnits describe the
%% units of each quantity in a standard way, allowing a client to do the conversion if the underlying transformation 
%% model is known. This is the case for simple operations such as converting light wavelengths into
%% frequencies, but also for more complex ones such as coordinate conversions or converting
%% magnitudes into fluxes.

%The former is the domain of the
%PhotometryDataModel to be issued in a separate Data Model effort. See  
%(\violet{\footnotesize{\url{http://www.ivoa.net/cgi-bin/twiki/bin/view/IVOA/PhotometryDataModel}}}) for more information and current developments.\\
%Excellent libraries \eg \textbf{AST} in C \cite{AST}
% (\violet{\footnotesize{\url{http://www.starlink.ac.uk/\~dsb/ast/ast.html}}})\\,
%or \textbf{SLALIB} for Java, already exist for the latter.
%\violet{\footnotesize{\url{http.www.starlink.rl.ac.uk/star/docs/sun67.htx/sun67.html}}}\\

This Recommendation describes only isolated units, and not arrays,
records or other combinations of units.  Thus, it does not cover
declaring an RA-Dec pair, for example. Several VO protocols require
embedding complex objects into result table and give string
serializations for those: geometries in TAP results are the most
common example.  This specification does not cover this situation,
although we hope that where individual unit strings are covered in
such instances, their syntax will conform to this specification.

\section{Proposal for VOUnits\label{sec:proposal}}

%% By systematically comparing the four reference schemes described above (details are given in Appendix~\ref{appx:comparisons}), 
The rules for VOUnits are defined in this section.
Various aspects are addressed:
\begin{itemize}
\item how the labels are encoded;
\item what base symbols are allowed and how they are spelled;
\item what prefixes are allowed and how they are used;
\item how symbols are combined.
\end{itemize}

A formal grammar summarizing these conventions is given
in \prettyref{appx:vougrammar}.

The text below is expected to be compatible with the prescriptions
of \citet{si-brochure}, except where noted.

\subsection{String representation and encoding\label{sec:encoding}}


VOUnits must be unit labels suitable for IVOA use (or other electronic manipulation),
and therefore need to be expressed unambiguously without encountering
encoding problems.  

Following the current usage in unit labels
(see \prettyref{tabx:comparUnitEncoding}), \emph{VOUnits must be
case-sensitive strings of chars consisting of printable ASCII
characters} (values coded 20 to 7E in hexadecimal).  The syntax
therefore excludes special characters such as \AA\ or \micro.

\subsection{Base units\label{sec:baseUnits}}

There is good agreement for the base symbols across the different schemes
(see \prettyref{tabx:comparUnitBase}).  VOUnits follow the same notations.

The VOUnits base symbols are listed in \prettyref{tab:voubase}

\begin{table}[ht]
\begin{center}
\def\arraystretch{1.2}
\begin{tabular}{|rl|rl|rl|rl|}\hline
\unit{m}&(metre)		&\unit{g}&(gramme) 	&\unit{J}&(joule)     	&\unit{Wb}&(weber)\\
\unit{s}&(second of time)	&\unit{rad}&(radian)    &\unit{W}&(watt) 	&\unit{T}&(tesla)\\
\unit{A}&(ampere)		&\unit{sr}&(steradian)  &\unit{C}&(coulomb)	&\unit{H}&(henry)\\
\unit{K}&(kelvin)		&\unit{Hz}&(hertz)      &\unit{V}&(volt) 	&\unit{lm}&(lumen)\\
\unit{mol}&(mole)		&\unit{N}&(newton)      &\unit{S}&(siemens)	&\unit{lx}&(lux)\\
\unit{cd}&(candela)		&\unit{Pa}&(pascal)     &\unit{F}&(farad)	&\unit{Ohm}&(ohm)\\\hline
\end{tabular}
\end{center}
\caption{\label{tab:voubase}VOUnits base units}
\end{table}

For masses, the SI unit is \unit{kg}. However, existing specifications
recommend not using scale factors with \unit{kg}, but attaching them
only to \unit{g} instead.
%% which makes the parsing potentially difficult. In VOUnits, only the 
%% \unit{g} base symbol is defined, and \unit{kg} is implicitly allowed
%% by using the 'k' scale factor with this base symbol. 
%% This is just a way to simplify the parsing, by simultaneously allowing scale factors for \unit{g} and
%% disallowing additional scale factors for \unit{kg}.

%% No use of the Ohm was found in astronomy archives, therefore the
%% possible non-compliance of the OGIP syntax was not considered
%% critical.

Recognising a known unit takes priority over parsing for prefixes.
Thus the string \unit{Pa} represents the Pascal, and not the
peta-year, and the string \unit{mol} will always be the mole, and
never a milli-`ol', for some unknown unit~`ol'.

\subsection{Known units}
\label{sec:knownunits}

In \prettyref{tabx:knownunits}, we indicate the `known units' for each of the
described syntaxes, which go beyond the physically motivated set of
base units.
There are a few units (namely `\unit{angstrom} or \unit{Angstrom}', 
\unit{pix} or \unit{pixel}', `\unit{ph} or \unit{photon}' and `\unit{a} or \unit{yr}') for
which there are recognised alternatives in some syntaxes, and in these
cases `p' marks the preferred one.

\emph{We recommend that unrecognised units be accepted by parsers},
as long as they are parsed giving preference to the syntaxes and prefixes described here.
Thus the string \unit{furlong/week} should parse successfully (though
perhaps with suitably prominent warnings) as the femto-`urlong' per week.

\begin{table}
\hbox to \textwidth{\hss
\catcode`\%=11
\begin{tabular}{rlcccc|rlcccc}
\emph{unit}&\emph{description}&\emph{fits}&\emph{ogip}&\emph{cds}&\emph{vou}&
\emph{unit}&\emph{description}&\emph{fits}&\emph{ogip}&\emph{cds}&\emph{vou}\\
\iffalse
% Generated from unity-grammars.zip/known-units.csv
%
% These are the grammars from https://bitbucket/nxg/unity
%     revision: b49aceb6d7f0
%     date: 2013-07-24 16:42 +0100
%     tags: tip
\fi
%&percent&&&$\cdot$& & K&kelvin&s&s&s&s\\
A&ampere&s&s&s&s & lm&lumen&s&s&s&s\\
a&julian year&s&&s&s & lx&lux&s&s&s&s\\
adu&ADU&$\cdot$&&&s & lyr&light year&$\cdot$&$\cdot$&&s\\
Angstrom&angstrom&d&&$\cdot$&dp & m&meter&s&s&s&s\\
angstrom&angstrom&&$\cdot$&&d & mag&magnitudes&s&$\cdot$&s&s\\
arcmin&arc minute&$\cdot$&$\cdot$&$\cdot$&s & mas&milliarcsecond&$\cdot$&&$\cdot$&$\cdot$\\
arcsec&arc second&$\cdot$&$\cdot$&s&s & min&minute (time)&$\cdot$&$\cdot$&$\cdot$&s\\
AU&astro'l unit&$\cdot$&$\cdot$&$\cdot$&s & mol&mole&s&s&s&s\\
barn&barn&sd&$\cdot$&s&sd & N&newton&s&s&s&s\\
beam&beam&$\cdot$&&&s & Ohm&ohm&s&&s&s\\
bin&bin&$\cdot$&$\cdot$&&s & ohm&ohm&&s&&\\
bit&bit&s&&s&sb & Pa&pascal&s&s&s&s\\
byte&byte&sp&$\cdot$&s&sbp & pc&parsec&s&s&s&s\\
B&byte&&&&sb & ph&photon&$\cdot$&&&s\\
C&coulomb&s&s&s&s & photon&photon&p&$\cdot$&&sp\\
cd&candela&s&s&s&s & pix&pixel&$\cdot$&&$\cdot$&s\\
chan&channel&$\cdot$&$\cdot$&&s & pixel&pixel&p&$\cdot$&&sp\\
count&number&$\cdot$&$\cdot$&&sp & R&rayleigh&s&&&s\\
Crab&crab&&s&& & rad&radian&s&s&s&s\\
ct&number&$\cdot$&&$\cdot$&s & Ry&rydberg&$\cdot$&&s&s\\
d&day&$\cdot$&$\cdot$&$\cdot$&s & s&second (time)&s&s&s&s\\
dB&decibel&&&&$\cdot$ & S&siemens&s&s&s&s\\
D&debye&$\cdot$&&$\cdot$&s & solLum&luminosity&$\cdot$&&$\cdot$&s\\
deg&degree (angle)&$\cdot$&$\cdot$&$\cdot$&s & solMass&solar mass&$\cdot$&&$\cdot$&s\\
erg&erg&d&$\cdot$&&sd & solRad&solar radius&$\cdot$&&$\cdot$&s\\
eV&electron volt&s&s&s&s & sr&steradian&s&s&s&s\\
F&farad&s&s&s&s & T&tesla&s&s&s&s\\
g&gramme&s&s&s&s & u&AMU&$\cdot$&&&s\\
G&gauss&sd&$\cdot$&&sd & V&volt&s&s&s&s\\
H&henry&s&s&s&s & voxel&voxel&$\cdot$&$\cdot$&&s\\
h&hour&$\cdot$&$\cdot$&$\cdot$&s & W&watt&s&s&s&s\\
Hz&hertz&s&s&s&s & Wb&weber&s&s&s&s\\
J&joule&s&s&s&s & yr&julian year&sp&$\cdot$&sp&sp\\
Jy&jansky&s&s&s&s\\

\end{tabular}
\hss}
\caption[Known units in the various syntaxes]
{\label{tabx:knownunits}Known units in the various syntaxes.
In the table, a `$\cdot$' indicates that the unit is recognised in that syntax,
an~`s' that it it additionally permitted to have SI prefixes,
a~`b' that it is permitted to have binary prefixes,
and a `d' that it is recognised but deprecated.
For those units which have alternative symbols, a~`p' indicates the
preferred one.}
\end{table}

\subsection{Binary units}

The symbol~`b' is sometimes used for `bits', but this is the SI symbol
for `barn', and this Recommendation aligns with the SI standard in
this respect.

\citet[item 13-9.c]{std:iec80000-13} notes that the term `byte'
`has been used for numbers of bits other than eight' in the past, but
that it should now always be used for eight-bit bytes; we recommend
the same interpretation here.  The same source notes the theoretical
confusion between the symbol \units{B} for `byte' and for `Bel'.  We
believe it would be perverse in our present context to recommend
against using `B' for byte, and so resolve this in this Recommendation
in favour of `byte' by mandating that \unit{B} is an
acceptable symbol for `byte', that the \unit{dB} is an
unprefixable special-case unit, and by implication that the `dB'
should not be interpreted as a tenth of a byte.

\subsection{Scale factors\label{sec:scaleFactors}}

Units can be prefixed by any of the 20 SI scale factors, or the eight
binary scale factors.
The SI scale factors -- provided in \prettyref{tab:vouscalefactors}a --
are the same as those of \citet{si-brochure},
of \citet[\S6.5.4]{std:iso80000-1},
and of \citet[Table~5]{pence10}
(see also \prettyref{tabx:comparUnitScale} for further comparisons).
%\medskip
\begin{table}
\def\arraystretch{1.2}
\begin{center}
\def\pfx#1#2{#1, $10^{#2}$}
\begin{tabular}{|rl|rl|}\hline
\unit{Y}&\pfx{yotta}{24}&
  \unit{y}&\pfx{yocto}{-24}\\
\unit{Z}&\pfx{zetta}{21}&
  \unit{z}&\pfx{zepto}{-21}\\
\unit{E}&\pfx{exa}{18}&
  \unit{a}&\pfx{atto}{-18}\\
\unit{P}&\pfx{peta}{15}&
  \unit{f}&\pfx{femto}{-15}\\
\unit{T}&\pfx{tera}{12}&
  \unit{p}&\pfx{pico}{-12}\\
\unit{G}&\pfx{giga}{9}&
  \unit{n}&\pfx{nano}{-9}\\
\unit{M}&\pfx{mega}{6}&
  \unit{u}&\pfx{micro}{-6}\\
\unit{k}&\pfx{kilo}{3}&
  \unit{m}&\pfx{milli}{-3}\\
\unit{h}&\pfx{hecto}{2}&
  \unit{c}&\pfx{centi}{-2}\\
\unit{da}&\pfx{deca}{1}&
  \unit{d}&\pfx{deci}{-1}\\
\hline
\end{tabular}
\qquad
\def\pfx#1#2{#1, $2^{#2}$}
\begin{tabular}{|rl|}\hline
\unit{Ki}&\pfx{kibi}{10}\\
\unit{Mi}&\pfx{mebi}{20}\\
\unit{Gi}&\pfx{gibi}{30}\\
\unit{Ti}&\pfx{tebi}{40}\\
\unit{Pi}&\pfx{pebi}{50}\\
\unit{Ei}&\pfx{exbi}{60}\\
\unit{Zi}&\pfx{zebi}{70}\\
\unit{Yi}&\pfx{yobi}{80}\\
\hline
\end{tabular}
\end{center}
\caption[VOUnits prefixes]{\label{tab:vouscalefactors}VOUnits prefixes:
(a, left) decimal prefixes;
(b, right) binary prefixes}
\end{table}

Writers of unit strings must not use compound prefixes (that is,
more than one SI prefix). Prefixes are concatenated to the base
symbol without space, and cannot be used without a base symbol.

The SI prefixes of \prettyref{tab:vouscalefactors}a \emph{always refer to
multiples of 1000}, even when applied to binary units such as bit or
byte; this follows the stipulations (and clarifying note) of
\citet[\S3.1]{si-brochure}, and the proscription
of \citet[\S6.5.4]{std:iso80000-1}.
If data providers wish to use multiples of 1024 (ie, $2^{10}$) for
units such as bytes or bits, they should use the the binary prefixes
of \citet[\S4]{std:iec80000-13}, reproduced in \prettyref{tab:vouscalefactors}b
(these were originally specified in \citet{std:ieee1541-2002}).  

Note: The letter \unit{u} is used instead of the
\micro\ symbol to represent a factor of $10^{-6}$, 
following the character set defined in \prettyref{sec:encoding}.

\subsection{Astronomy symbols}

\prettyref{tabx:comparUnitAstro} lists symbols used in astronomy to
describe times, angles, distances and a few additional quantities.
The subset of these used by this specification are
listed in \prettyref{tab:vouadopted}.

\begin{table}[ht]
\begin{center}
\def\arraystretch{1.2}
\begin{tabular}{|rl|rl|rl|}\hline
\unit{min}&(minute of time)	&\unit{deg}&(degree of angle) 	&\unit{AU}&(astronomical unit)     	\\
\unit{h}&(hour of time)		&\unit{arcmin}&(arcminute)    	&\unit{pc}&(parsec) 	\\
\unit{d}&(day)			&\unit{arcsec}&(arcsecond)  	&\unit{eV}&(electron volt)	\\
\unit{a} or \unit{yr}&(year)	&\unit{mas}&(milliarcsecond)    &\unit{Jy}&(jansky) 	\\
\unit{u}&(atomic mass)		&      				&	\\\hline
\end{tabular}
\end{center}
\caption{\label{tab:vouadopted}Additional astronomy symbols}
\end{table}


Minutes, hours, and days of time must be represented in VOUnits by the
symbols \unit{min}, \unit{h} and \unit{d}; however the \unit{cd} is
the candela, not the centi-day.  The year can be expressed by
\unit{yr} (common practice),
or \unit{a},
as recommended by ISO \citep[Annex C]{std:iso80000-3}
and the IAU \citep[Table 6]{wilkins89}.
However peta-year must only be written \unit{Pyr},
to avoid the collision with the pascal, \unit{Pa}.

There are no VOUnit symbols for degrees celsius or century.
Temperatures are expressed in kelvin (\unit{K}),
and a century corresponds to \unit{ha} or \unit{hyr}.
Note that \emph{this is a mild deviation from the SI standard},
which states that the `hectare', with unit symbol \unit{ha},
is a `non-SI unit accepted for use' as a measure of land area~\citep[table~6]{si-brochure},
and which acknowledges neither `a' nor `yr' as a symbol for year.\footnote{If
large telescope arrays feel they must talk of attojoules per
hectare per century, for some reason, they're going to have to be
careful how they do so; it's probably best not to even think about atto-Henrys.}

The astronomical unit must be expressed in upper-case, \unit{AU},
in order to follow the general expectation of legacy practice,
and to avoid the (admittedly unlikely) confusion with atto-atomic mass.
\emph{This is a deviation} from both the SI recommendation of
`ua'~\citep[Table 7]{si-brochure} and the IAU's recommendation of `au'~\citep{iau12}.%
\footnote{If you feel a burning desire to write about micro-years or
atto atomic-mass, this document is not the place you need to look
for help.}

A few symbols which might theoretically be ambiguous are listed in \prettyref{tab:ambiguous},
with their correct VOUnit interpretation and what they do not mean.


\begin{table}[bht]
\begin{center}
\begin{tabular}{lll}
\textbf{VOUnit}&\textbf{Correct interpretation}&\textbf{Incorrect}\\
\unit{Pa}&pascal&peta-year\\
\unit{ha}&hecto-year&hectare\\
\unit{cd}&candela&centi-day\\
\unit{au}&atto-atomic-mass&astronomical unit\\
\end{tabular}
\end{center}
\caption{\label{tab:ambiguous}Possibly ambiguous units}
\end{table}

\subsection{Other symbols}

\prettyref{tabx:comparUnitDeprecated} corresponds to Table~7 in the IAU document, and the IAU strongly
recommends no longer using these units. 
Data producers are strongly advised to prefer the equivalent notation using symbols and prefixes listed in 
Tables~\ref{tabx:comparUnitBase}, \ref{tabx:comparUnitScale} and \ref{tabx:comparUnitAstro}. 

However, in order to be compatible with legacy metadata, VOUnit parsers should be able to interpret
symbols for \aa{}ngstr\"om, barn, erg and gauss, as below:
%in the sixth column of \prettyref{tabx:comparUnitDeprecated}.

\medskip

\begin{center}
\begin{tabular}{|c|}\hline
\def\arraystretch{1.2}
\unit{angstrom} or \unit{Angstrom}\\
\unit{erg}    	\\
\unit{barn}\\
\unit{G} (Gauss)    		\\\hline
\end{tabular}\\
\end{center}

\medskip

\prettyref{tabx:comparUnitOther} compares other miscellaneous symbols. 

The last set of VOUnits symbols, derived from this comparison, is in
\prettyref{tab:voumisc}

%\medskip
\begin{table}[ht]
\begin{center}
\def\arraystretch{1.2}
\begin{tabular}{|l|l|p{3cm}|l|}\hline
\unit{mag} (magnitude)		&\unit{pix}  or \unit{pixel} 	&\unit{solMass} (solar mass)     &\unit{R} (rayleigh) \\
\unit{Ry} (rydberg)		&\unit{voxel}    		&\unit{solLum} (solar luminosity)&\unit{chan} (channel) 	\\
\unit{lyr} (light year)		&\unit{bit}   			&\unit{solRad} (solar radius)	&\unit{bin} \\
\unit{ct} or \unit{count}	&\unit{byte} (8 bits)   	&\unit{Sun} (relative to the Sun, e.g. abundances)&\unit{beam} 	\\
\unit{ph} or \unit{photon} 	&\unit{adu}     		&\unit{D} (Debye)	&\unit{?} (unknown)\\\hline
\end{tabular}
\end{center}
\caption{\label{tab:voumisc}Miscellaneous VOUnits}
\end{table}

It can be noted that some of the units listed in \prettyref{tabx:comparUnitOther} are 
questionable. They arise in fact from a need to describe quantities, when the only
piece of metadata available is the unit label. Count, photon, pixel, bin, voxel, bit,
byte are concepts, just as apple or banana. The associated quantities could be fully
described with a UCD, a value and a void unit label.

It is possible to count a number of bananas, or to express a distance measured in
bananas, but this does not make a banana a reference unit.

The FITS document provides the most general description of all the compared schemes, 
and VOUnits adopts similar definitions, in order to acknowledge at best legacy metadata.
The VOUnits symbol for magnitudes is \unit{mag}.
The symbol \unit{Sun} is used to express ratios relative to solar values, for example
abundances or metallicities.
Note that all symbols like \unit{count}, \unit{photon}, \unit{pixel} are always used
in lower case and singular form.

The decibel, \unit{dB} is listed in the SI specification \citep[Table
8]{si-brochure} amongst a set of `other non-SI units', and mentioned
by \citet[\S0.5]{std:iso80000-3} in a `Remark on logarithmic
quantities'.  Here, we recommend that the \unit{dB} is parsed as a
unit by itself -- as opposed to being parsed as the prefix~`d' qualifying the unit `Bel' --
and that neither it nor the Bel may be used with other scaling prefixes.

The recommended way to indicate that there is no unit (for example a quantity that
is a character string, or unitless) is to use an empty string rather than blanks
or dashes.

The question mark (\unit{?}) is reserved for cases when the unit is unknown.


\subsection{Mathematical expressions containing symbols}

\prettyref{tabx:comparUnitCombine} summarizes how, 
in the various existing syntaxes,
mathematical operations can be applied on unit symbols for
exponentiation, multiplication, division, and other computations.

The combination rules are where the largest discrepancies between the
different schemes appear. The FITS document discusses the problem of
trying to best accommodate the existing schemes
\cite[\S4.3.1]{pence10}, without really resolving the problem.
\label{sec:fitsquote}
%% \begin{quote}
%% Two or more base units strings (called str1 and str2 (...)) may be combined using
%% the restricted set of (explicit or implicit) operators that provide
%% for multiplication, division, exponentiation, raising arguments
%% to powers, or taking the logarithm or square-root of an
%% argument. Note that functions such as log actually require dimensionless
%% arguments, so that $\log(\unit{Hz})$, for example, actually
%% means $\log(x/1 \unit{Hz})$. The final units string is the compound
%% string, or a compound of compounds, preceded by an
%% optional numeric multiplier of the form \texttt{10**k}, \texttt{10\^{}k}, or \texttt{10$\pm$k}
%% where~\texttt k is an integer, optionally surrounded by parentheses with
%% the sign character required in the third form in the absence of 
%% parentheses. Creators of FITS files are encouraged to use the
%% numeric multiplier only when the available standard scale factors
%% (\dots) will not suffice. Parentheses are used for symbol
%% grouping and are strongly recommended whenever the order
%% of operations might be subject to misinterpretation. A space
%% character implies multiplication which can also be conveyed explicitly
%% with an asterisk or a period. Therefore, although spaces
%% are allowed as symbol separators, their use is discouraged. Note
%% that, per IAU convention, case is significant throughout. The IAU
%% style manual forbids the use of more than one solidus (\unit{/}) character
%% in a units string. However, since normal mathematical precedence
%% rules apply in this context, more than one solidus may be
%% used but is discouraged.
%%
%% A unit raised to a power is indicated by the unit string followed,
%% with no intervening spaces, by the optional symbols \texttt{**}
%% or \texttt{\^{}} followed by the power given as a numeric expression (\dots). 
%% The power may be a simple integer, with or
%% without sign, optionally surrounded by parentheses. It may also
%% be a decimal number (\eg 1.5, 0.5) or a ratio of two integers
%% (\eg 7/9), with or without sign, which must be surrounded by
%% parentheses. Thus meters squared may be indicated by m**(2),
%% \texttt{m**+2}, \texttt{m+2}, \texttt{m2}, \texttt{m\^{}2}, \texttt{m\^{}(+2)}, etc. and per meter cubed may be
%% indicated by \texttt{m**-3}, \texttt{m-3}, \texttt{m\^{}(-3)}, \texttt{/m3}, and so forth. Meters to
%% the three-halves power may be indicated by \texttt{m(1.5)}, \texttt{m\^{}(1.5)},
%% \texttt{m**(1.5)}, \texttt{m(3/2)}, \texttt{m**(3/2)}, and \texttt{m\^{}(3/2)}, but not by \texttt{m\^{}3/2}
%% or \texttt{m1.5}.~\cite[\S4.3.1]{pence10}
%% \end{quote}
%%
This and other ambiguities are discussed in the detailed syntaxes of \prettyref{appx:grammar}.

VOUnits follow a subset of the FITS rules,
as summarized in \prettyref{tab:VOUnitCombine},
and can in addition be preceded by an optional numeric scaling factor,
which must be a power of ten.
%% (expressed as a floating number with the optional
%% exponent being expressed as in FITS, with a multiplication symbol).
%% % XXX Aren't these quantities, rather than units?? 
%% Examples of valid VOUnits are \unit{2.54cm}, \unit{10+8m}, \unit{3.45 10**(-4)Jy}. 
Scale factors from \prettyref{tabx:comparUnitScale} should however be
preferred and used whenever possible.
%\emph{Note (NG): these appear to be `quantities' rather than `units',
%  in the terminology of \prettyref{sec:notations} -- resolve
%  during/after RFC period.}

\begin{table}%[ht]
\begin{center}
\def\arraystretch{1.2}
\begin{tabular}{|r|l|}
\hline
%\unit{str1 str2} & Multiplication (discouraged -- see text)\\
%\unit{str1*str2} & Multiplication \\
\unit{str1.str2} & Multiplication \\
\unit{str1/str2} & Division \\
\unit{str1**expr} & Raised to the power expr \\
%\unit{str1\^{}expr} & Raised to the power expr \\
%\unit{str1expr} & Raised to the power expr \\
\unit{fn(str1)} & Function applied to a unit string\\
%% \unit{log(str1)} & Common Logarithm (to base 10) \\
%% \unit{ln(str1)} & Natural Logarithm \\
%% \unit{exp(str1)} & Exponential (e$^\mathrm{str1}$) \\
%% \unit{sqrt(str1)} & Square root \\
\hline
\end{tabular}
\end{center}
 \caption[Combination rules and mathematical expressions for VOUnits]
{\label{tab:VOUnitCombine}Combination rules and mathematical expressions for VOUnits.
See \prettyref{appx:vougrammar} for the complete grammar.}
\end{table}

As illustrated in \prettyref{tab:VOUnitCombine}, units may include a
limited set of functional dependencies on other units.  The set of
functions recognised within VOUnits is the same as the set recommended
by FITS, and listed in \prettyref{tab:functions}.  As with
unrecognised units,
\emph{we recommend that parsers accept unrecognised functions without error},
even if they deprecate them at some later processing stage.
\begin{table}%[ht]
\begin{center}
\def\arraystretch{1.2}
\begin{tabular}{|r|l|}
\hline
\unit{log(str1)} & Common Logarithm (to base 10) \\
\unit{ln(str1)} & Natural Logarithm \\
\unit{exp(str1)} & Exponential (e$^{\mathrm{str1}}$) \\
\unit{sqrt(str1)} & Square root \\
\hline
\end{tabular}
\end{center}
\caption{\label{tab:functions}Functions of units.}
\end{table}


%\subsection{Quantities}
%\label{sec:quantities}
%
%A quantity, \eg a measurement of a physical value like the speed of
%light, has a value (2.998 10+5), a ucd (phys.veloc), units (km.s-1)
%and is coded using a numerical type (real).
%
%Some quantities are also reused as units. Many units are expressed, or
%converted, in terms of physical constants such as the speed of light,
%\begin{itemize}
%	\item \texttt{$c=$~2.998 10+8~m.s-1;} 
%	\item Boltzman's constant, \texttt{$K_{\mathrm{B}}=$1.38065~10-23}
%	\item \texttt{1 AU $=$ 1.499 10+11 m.}
%\end{itemize}
%
% Many of these are used as units in their own right, \eg  velocities may be expressed as a
%fraction or multiple of c, but c is also used to convert between
%wavelength and frequency, etc. These are combinations of units with
%scaling factors applied, and so can be treated in the same way as any
%other compound unit \eg  the \texttt{Jy} (\texttt{10-26 W.m-2.Hz-1}) .
%
%We need to ensure that we are consistent with the IVOA Quantity model,
%where appropriate.

\subsection{Remarks and good practices}

VOUnits, as described in this document, provide a string serialization of
unit labels compatible with the two principal
means of representation expected in the VO:
\begin{itemize}
\item as an attribute string in a VOTable document: \verb|unit="..."|
\item as a unit element in an XML serialization of some data model: \verb|<unit>...</unit>|
\end{itemize}

Avoiding the use of special characters (\AA, ', ", ...) reduces the
complexity of using VOUnits in query languages or other programming environments.

Even if enclosing VOUnits with single or double quotes should be sufficient to
facilitate parsing of a general query containing VOUnits, avoiding the use of
white spaces for multiplication should
make the parsing even easier, with the unit label being a single
`word'.
%Therefore the notation in the first line of \prettyref{tab:VOUnitCombine} is discouraged.


Complex data formats are not addressed by VOUnits. While possibly convenient for
humans, sexagesimal coordinates or calendar dates expressed in ISO 8601 are
quantities represented in a complex format, encoded as strings, and as such the
corresponding VOUnit should be an empty string. Expressions such as ``d:m:s'' or
``ISO 8601'' are not valid VOUnits. This should not be a problem, as existing VO 
standards already recommend that coordinates be expressed in decimal degrees.

One can also remark that some quantities like the Modified Julian Date (MJD) are 
not recognized VOUnits. As described in \prettyref{sec:notations}, the quantity MJD can be 
seen as a concept (described by the appropriate UCD or utype), and the corresponding
value will most likely be expressed in days, so the VOUnit will be \unit{d}. There is 
no need to overload VOUnits to incorporate the description of concepts themselves.

A summary of the various symbols is given in the last part of Appendix~\ref{appx:comparisons}.

The notion of unit conversion and quantity manipulation is discussed in
\prettyref{sec:conversion}.


\section{Use cases and applications\label{sec:useCase}}

\subsection{Unit parsing}

The rules defined in \prettyref{sec:proposal} allow us to build VOUnit parsers.
Several services can be built on top of a VOUnit parser:

\begin{enumerate}
\item Validation. A service checking that a VOUnit is well written. The output
of such a service can have different levels: fully valid unit; valid syntax, but
not the preferred one (\eg  use of deprecated symbols); parsing error. 
\item Explanation. A service returning a plain-text explanation of the unit label.
\item Typesetting. A service returning an equivalent of the unit label suitable for inclusion in
a \LaTeX\ or HTML document.
\item Dimensional equation. As described by \citet{osuna05}, VOUnits can be translated
into a dimensional equation, allowing to build up conversions methods from one string 
representation to another one (see also \prettyref{sec:conversion}). 
\end{enumerate}

\subsection{Libraries\label{sec:libraries}}

There are a few existing libraries able to interpret unit labels.
In all cases,
some software effort is required if they are to be used in translating
between data provider unit labels, and those to be adopted by
the IVOA for internal use.

One of the most widely-used specialised
astronomical libraries is AST which includes a unit conversion
facility attached to astronomical coordinate systems \citep{berry12}.

Another library has been developed at
CDS\footnote{\url{http://cds.u-strasbg.fr/resources/doku.php?id=units}},
and can be tested online\footnote{\url{http://cdsweb.u-strasbg.fr/cgi-bin/Unit}}. This library covers all
the symbols and notations defined in the standard for astronomical catalogues \citep[\S3.2]{cds00}, as well as
additional symbols and notations.

The Unity library\footnote{\url{https://bitbucket.org/nxg/unity}} is a new
standalone library intended to parse VOUnits, OGIP, StdCats and FITS
formats; it was used as a vehicle for developing and testing the grammars and
ideas for this present document.  It provides yacc-style grammars for
the various syntaxes, as well as implementing them in parsers written
in Java and~C.  The grammars of \prettyref{appx:grammar} are extracted
from the Unity distribution.

\subsection{Unit conversion and quantity transformation\label{sec:conversion}}

Unit conversion is the simple task of converting a quantity expressed in a given unit into a different unit, while
the concept remains the same. For example, converting a distance in \unit{pc} into a distance in \unit{AU} or \unit{km}.
Or converting a flux from \unit{mJy} to \unit{W.m-2.Hz-1}. This is rather easy with existing libraries, using dimensional
analysis or SI units as a reference.

Quantity transformation consists in deriving a new quantity from one or several original
quantities. It is more complex, because it requires to have a precise model 
(a simple equation in simple cases) for computing the transformation. The model involves
quantities, each described with a UCD or utype, value and VOUnit. Some of the quantities
involved might be physical constants (\eg  Boltzmann's constant $k_{\mathrm{B}}$).

Examples of such transformations can be:
\begin{itemize}
\item linear unit conversion: a distance is measured in \unit{pixel} in an image, and needs to be transformed in
the corresponding angular separation in \unit{arcsec}. This can be done if the quantity representing the pixel
scale is given, with its value and a compatible unit like \unit{deg/pixel}.
\item converting a photon wavelength in the corresponding photon energy or frequency.
\item deriving the flux for a given photon emission rate (in \texttt{W}) from Planck's
constant (\texttt{6.63 $10^{-34}$ J.s}), the radiation frequency (in \texttt{GHz}), and the
number of photons emitted per second.
\item transforming a magnitude into a flux, as needed for SED building.
\end{itemize}

VOUnits can help in quantity transformation if all quantities are qualified with proper VOUnits.

\subsection{Query languages}

Including VOUnits in queries is not an easy task. Some guidelines were defined in the
reflexion on ADQL.

\begin{enumerate}
\item All data providers should be encouraged to supply units for each column
of a table. Columns should also have associated UCDs, so that quantities can be
properly identified.

%In most published tables in Astronomical journals and Vizier server as well, unitless values are
%represented by "---". This could be adopted for the VO convention as well.
\item The IVOA needs to provide a parser to relate the native units to the standard IVOA
labels (in this context, the `native units' are the units of the
underlying database table or metadata).   

\item
The default response to a query which does not specify units, will be
in the native units of the table. 
%\emph{We recommand that the output units will be labelled using the IVOA standard label ???}

\item
Where queries involve combining or otherwise operating on the content
of columns to produce an output column with modified units, we can
provide libraries and a parser to assist in assigning and checking a
new unit, and attach this to the returned values via the SQL CAST
operator. 
This is implemented already in database related applications such as 
Saada\footnote{\url{http://saada.unistra.fr/}}, for instance. 
If any column used in responding to a query lacks a necessary unit, the output
involving that column will be unitless.

\item
If the user wants to change the output units with respect to the table
units, this could be done by specifying the units in the initial
SELECT statement. There are several issues to consider: 
	\begin{enumerate}
	\item Does the user also need to include the conversion expression, or does the unit
parser take care of that?  
	\item Can the user use this to assign units (based on prior knowledge) to output from a 
column lacking a unit?
	\end{enumerate}
\end{enumerate} 


\subsection{Broader use in the VO}

\begin{figure}[thb]
  % Requires \usepackage{graphicx}
  \includegraphics[width=\textwidth]{./units2.jpg}
  \caption{This shows the levels at which conversions might be done.
\textcolor{blue}{Blue arrows}: At the point where an astronomer or
  data provider submits input to the VO, we should provide tools to
  ensure that units are labeled consistently according to VOUnits. 
  This implies that a units parsing step is included prior to metadata ingestion into the VO.
\brown{Brown arrows}: Conversions required to supply results to
  the user in specified or reasonable units \eg  \texttt{J.s-1} to \texttt{W}, are done where and when they are required.}
  \label{fig:units2}
\end{figure}

Different VO entities require and consume metadata with units attached like registries, 
applications and interoperate via protocols. \prettyref{fig:units2} illustrates the places where the IVOA
could intervene to ensure consistent use of units.


\newpage

\appendix

\section{Current use of units}
\label{appx:current}

Many other projects have already produced lists of preferred
representations of units. Those most commonly used in
astronomy are described in this section. 

The four first schemes described below are used as references for the
comparison tables presented later in this document.

\subsection{IAU 1989\label{appx:IAU}}

In the section 5.1 of its Style Manual, the IAU gives a set
of recommendations for representing units in publications \citep{wilkins89}. This document
therefore provides useful reference guidelines, but is not directly
applicable to VOUnits because the recommendations are more intended
for correct typesetting in journals than for standardized metadata exchange.
The IAU style will be summarized in the second column of the comparison tables.

\subsection{OGIP 1993}

NASA has defined a list of character strings specifying the basic physical units 
used within OGIP (Office of Guest Investigator Programs) FITS files \citep{george95}. Rules and guidelines on the construction 
of compound units are also outlined. 

HEASARC datasets follow these conventions, presented in the third column
of the comparison tables.

\subsection{Standards for astronomical catalogues}

The conventions adopted at CDS are summarized in the Standards for Astronomical 
Catalogues, Version 2.0 \citep[\S3.2]{cds00}. They are presented in the fourth column
of the comparison tables.

\subsection{FITS 2010}

In Section 4.3 of the reference FITS paper, \citet{pence10} describe how unit strings are to be expressed in
FITS files. The recommendations are presented in the fifth column
of the comparison tables.

\subsection{Other usages}

\begin{itemize}
\item
\violet{\footnotesize{\url{http://arxiv.org/pdf/astro-ph/0511616}}}\\ 
Dimensional Analysis applied to spectrum handling in VO context~\citep{osuna05}
offers a mathematical framework to guess and recompute
SI units for any quantity in astronomy.
\item
\violet{\footnotesize{\url{http://www.mel.nist.gov/msid/sima/07_ndml.htm}}}\\
NIST (National Institute of Standards \& Technology) project
UnitsXML builds up an XML representation of units at the granularity
level of a simple symbol string
\item \violet{\footnotesize{\url{https://jsr-275.dev.java.net/}}}\\
JAVA JSR-275 specifies Java packages for the programmatic
handling of physical quantities and their expression as numbers of
units.
\item  \texttt{aips++}
\violet{\footnotesize{\url{http://aips2.nrao.edu/docs/aips++.html}}} and\\
 \texttt{casacore} \violet{\footnotesize{\url{http://code.google.com/p/casacore/}}}\\ contain modules handling units and
quantities with high precision. The packages are mainly in use for
radio astronomy but are designed to be modular and adaptable.  (NB
contrary to the statement on the casacore link, aips++ is still very much in
use as the toolkit behind the {\sc casa} package.)
%\item  IAU SOFA
%\violet{\footnotesize{\url{http://www.iau-sofa.rl.ac.uk/}}} and\\  
%USNO NOVAS
%\violet{\footnotesize{\url{http://aa.usno.navy.mil/software/novas/novas_info.php}}}\\
%implement the IAU 2000 recommendations.
\end{itemize}

\section{Comparison of existing schemes and VOUnits\label{appx:comparisons}}

%\subsection{String representation and encoding}

%    \setlength\LTcapwidth{22cm}
\begin{table}[ht]
%\begin{longtable}[th]{|p{0.2\linewidth}|p{0.15\linewidth}|p{0.12\linewidth}|p{0.12\linewidth}|p{0.12\linewidth}|p{0.15\linewidth}|}
  \begin{tabular}{|p{0.2\linewidth}|p{0.15\linewidth}|p{0.1\linewidth}|p{0.1\linewidth}|p{0.15\linewidth}|p{0.15\linewidth}|}
\hline
    & IAU & OGIP  & StdCats & FITS  & VOUnits\\\hline
    Units are strings of chars &  & YES &  & YES & YES\\\hline
    Case sensitive & YES & YES & YES & YES & YES\\\hline
    Character set &  &  & No spaces & ASCII text & ASCII printable\\\hline
\end{tabular}
  \caption{Comparison of string representation and encoding.}
  \label{tabx:comparUnitEncoding}
%\end{longtable}
\end{table}

%\subsection{Base units}

\begin{table}[ht]
\begin{tabular}{|p{0.2\linewidth}|p{0.15\linewidth}|p{0.12\linewidth}|p{0.12\linewidth}|p{0.12\linewidth}|p{0.15\linewidth}|}
%\begin{longtable}[th]{|p{0.2\linewidth}|p{0.15\linewidth}|p{0.12\linewidth}|p{0.12\linewidth}|p{0.12\linewidth}|p{0.15\linewidth}|}
\hline
    & IAU & OGIP  & StdCats & FITS  & VOUnits\\\hline
    %\multicolumn{6}{|c|}{Base units} \\\hline
    The 6+1 base SI units\raggedright & \multicolumn{4}{c|}{\unit{m, s, A, K, mol, cd}} & idem \\
    \cline{2-6}
    (use \unit{s}, not sec, for seconds)\raggedright & unit is \unit{kg}, but use \unit{g} with prefixes\raggedright & \unit{kg} & \unit{g} & \unit{kg}, note that \unit{g} is allowed & \unit{g}\\
     %& \multicolumn{4}{c|}{} &  \\
     \hline
    Dimensionless planar and solid angle\raggedright & \multicolumn{4}{c|}{\unit{rad, sr}} & idem \\
    \cline{2-5}
    &  &  &  & \unit{deg} preferred for decimal angles & \\\hline
    Derived units with symbols\raggedright & \multicolumn{4}{c|}{\unit{Hz, N, Pa,
      J, W, C, V, S, F, Wb, T, H, lm, lx}} & idem \\
     & \unit{$\Omega$} & \unit{ohm} & \unit{Ohm} & \unit{Ohm} & \unit{Ohm}\\\hline
%     &  &  &  &  & \\\hline
\end{tabular}
  \caption{Comparison of base units.}
  \label{tabx:comparUnitBase}
%\end{longtable}
\end{table}

%\subsection{Scale factors}

\begin{table}[ht]
\begin{tabular}{|p{0.2\linewidth}|p{0.15\linewidth}|p{0.12\linewidth}|p{0.12\linewidth}|p{0.12\linewidth}|p{0.15\linewidth}|}
%\begin{longtable}[th]{|p{0.2\linewidth}|p{0.15\linewidth}|p{0.12\linewidth}|p{0.12\linewidth}|p{0.12\linewidth}|p{0.15\linewidth}|}
\hline
    & IAU & OGIP & StdCats sec.~3.2.3 & FITS & VOUnits\\\hline
    Scale factors,   & \multicolumn{4}{c|}{\unit{d, c, m, n, p, f, a}} & idem \\
    (multiple) & \multicolumn{4}{c|}{\unit{da, h, k, M, G, T, P, E}}  & \\
    prefixes & \unit{\micro} & \multicolumn{3}{c|}{\unit{u}} & \unit{u}\\
     &  & \multicolumn{3}{c|}{\unit{z, y, Z, Y}} & \unit{z, y, Z, Y}\\\hline
    % & Tab. 5 & Tab. 3 & 3.2.3 & Tab. 5 & \\\hline
    Prefix--symbol concatenation & no space, regarded as single symbol\raggedright & no space, regarded as a single unit string\raggedright & no space & no space (implicit)\raggedright & no space\\\hline
    Prefix-able symbols  & Not \unit{kg}: use \unit{g}\raggedright & All units above, and \unit{eV, pc, Jy, Crab} Only \unit{mCrab} allowed\raggedright & all & all & all (except \unit{P} for \unit{a})\\\hline
    Use compound prefixes & should not & should never & must not & must not & must not\\\hline
\end{tabular}
  \caption{Comparison of scale factors.}
  \label{tabx:comparUnitScale}
%\end{longtable}
\end{table}

%\clearpage

%\subsection{Astronomy symbols}

%\clearpage

\begin{table}[ht]
\begin{tabular}{|p{0.2\linewidth}|p{0.15\linewidth}|p{0.12\linewidth}|p{0.12\linewidth}|p{0.12\linewidth}|p{0.15\linewidth}|}
%\begin{longtable}[th]{|p{0.2\linewidth}|p{0.15\linewidth}|p{0.12\linewidth}|p{0.12\linewidth}|p{0.12\linewidth}|p{0.15\linewidth}|}
\hline
    & IAU & OGIP  & StdCats & FITS  & VOUnits\\\hline
    %\multicolumn{6}{|c|}{Astronomy-related units} \\\hline
    minute & \unit{min, $^\mathrm{m}$} & \unit{min} & \unit{min} & \unit{min} & \unit{min}\\\hline
    hour & \unit{h, $^\mathrm{h}$} & \unit{h} & \unit{h} & \unit{h} & \unit{h}\\\hline
    day & \unit{d, $^\mathrm{d}$} & \unit{d} & \unit{d} & \unit{d} & \unit{d}\\\hline
    year & \unit{a} & \unit{yr} & \unit{a, yr} & \unit{a, yr} Pa (peta a) forbidden& like FITS\\\hline
    arcsecond & \unit{''} & \unit{arcsec} & \unit{arcsec} & \unit{arcsec} & \unit{arcsec}\\\hline
    arcminute & \unit{'} & \unit{arcmin} & \unit{arcmin} & \unit{arcmin} & \unit{arcmin}\\\hline
    degree (angle) & \unit{$^\circ$} & \unit{deg} & \unit{deg} & \unit{deg} & \unit{deg}\\\hline
    milliarcsecond & \unit{mas} (use \unit{nrad}!)\raggedright &  & \unit{mas} & \unit{mas} & \unit{mas}\\\hline
    microarcsec &  &  & \unit{uarcsec} &  & no dedicated symbol, use
    \unit{uarcsec}\\\hline
    cycle & \unit{c, $^\mathrm{c}$} &  &  &  & not used\\\hline
    astronomical unit & \unit{au} & \unit{AU} & \unit{AU} & \unit{AU} & \unit{AU}\\\hline
    parsec & \multicolumn{4}{c|}{\unit{pc}} & \unit{pc}\\\hline
    atomic mass & \unit{u} &  &  & \unit{u} & \unit{u}\\\hline
    electron volt & \multicolumn{4}{c|}{\unit{eV}} & \unit{eV}\\\hline
    jansky & \multicolumn{4}{c|}{\unit{Jy}} & \unit{Jy}\\\hline
    celsius degree & \unit{$^\circ$C} for meteorology, other use \unit{K}\raggedright&  &  &  & not used\\\hline
    century & ha, cy should not be used &  &  &  & no dedicated symbol, use \unit{ha} or \unit{hyr}\\\hline
\end{tabular}
  \caption{Comparison of astronomy-related units.}
  \label{tabx:comparUnitAstro}
%\end{longtable}
\end{table}
%\clearpage

%\subsection{Other symbols}

\begin{table}[ht]
\begin{tabular}{|p{0.2\linewidth}|p{0.15\linewidth}|p{0.12\linewidth}|p{0.12\linewidth}|p{0.12\linewidth}|p{0.15\linewidth}|}
%\begin{longtable}[th]{|p{0.2\linewidth}|p{0.15\linewidth}|p{0.12\linewidth}|p{0.12\linewidth}|p{0.12\linewidth}|p{0.15\linewidth}|}
\hline
    & IAU & OGIP  & StdCats & FITS  & VOUnits\\\hline
    %\multicolumn{6}{|c|}{IAU (Table 7) strongly recommends to no longer use these} \\\hline
    \aa{}ngstr\"om & \unit{\AA} & \unit{angstrom} & 0.1nm & \unit{Angstrom} & \unit{angstrom}, \unit{Angstrom}\\\hline
    micron & \unit{\micro} &  &  &  & not used \\\hline
    fermi & no symbol &  &  &  & not used \\\hline
    barn & \unit{b} & \unit{barn} & \unit{barn} & \unit{barn} & \unit{barn}\\\hline
    cubic centimetre & \unit{cc} &  &  &  & no dedicated symbol\\\hline
    dyne & \unit{dyn} & \unit{} & \unit{} & \unit{} & not used \\\hline
    erg & \unit{erg} & \unit{erg} & No symbol. \unit{mW/m2} used for erg.cm-2.s-1 & \unit{erg} & \unit{erg} \\\hline
    calorie & \unit{cal} & \unit{} & \unit{} & \unit{} & not used \\\hline
    bar & \unit{bar} & \unit{} & \unit{} & \unit{} & not used \\\hline
    atmosphere & \unit{atm} & \unit{} & \unit{} & \unit{} & not used \\\hline
    gal & \unit{Gal} & \unit{} & \unit{} & \unit{} & not used \\\hline
    eotvos & \unit{E} & \unit{} & \unit{} & \unit{} & not used \\\hline
    gauss & \unit{G} & \unit{G} & \unit{} & \unit{G} & \unit{G} \\\hline
    gamma & \unit{$\gamma$} & \unit{} & \unit{} & \unit{} & not used \\\hline
    oersted & \unit{Oe} & \unit{} & \unit{} & \unit{} & not used \\\hline
    Imperial, non-metric & should not be used & \unit{} & \unit{} & \unit{} & not used \\\hline\hline
\end{tabular}
  \caption{Comparison of symbols deprecated by IAU.}
  \label{tabx:comparUnitDeprecated}
\end{table}
%\end{longtable}

\begin{table}[ht]
\begin{tabular}{|p{0.2\linewidth}|p{0.15\linewidth}|p{0.12\linewidth}|p{0.12\linewidth}|p{0.12\linewidth}|p{0.15\linewidth}|}
%\begin{longtable}[th]{|p{0.2\linewidth}|p{0.15\linewidth}|p{0.12\linewidth}|p{0.12\linewidth}|p{0.12\linewidth}|p{0.15\linewidth}|}
\hline
    & IAU & OGIP  & StdCats & FITS  & VOUnits\\\hline
    magnitude & \multicolumn{4}{c|}{\unit{mag}} & \unit{mag}\\\hline
    rydberg & \unit{} & \unit{} & \unit{Ry} & \unit{Ry} & \multirow{19}{0.15\linewidth}{same as FITS} \\\hline
    solar mass & \unit{$\mathrm{M}_\odot$} &  & \unit{solMass} & \unit{solMass} &\\\cline{1-5}
    solar luminosity & \unit{} & \unit{} & \unit{solLum} & \unit{solLum} &\\\cline{1-5}
    solar radius & \unit{} & \unit{} & \unit{solRad} & \unit{solRad} &\\\cline{1-5}
    light year & \unit{} & \unit{lyr} & \unit{} & \unit{lyr} &\\\cline{1-5}
    count & \unit{} & \unit{count} & \unit{ct} & \unit{ct, count} &\\\cline{1-5}
    photon & \unit{} & \unit{photon} & \unit{} & \unit{photon, ph} &\\\cline{1-5}
    rayleigh & \unit{} & \unit{} & \unit{} & \unit{R} &\\\cline{1-5}
    pixel & \unit{} & \unit{pixel} & \unit{pix} & \unit{pix, pixel} &\\\cline{1-5}
    debye & \unit{} & \unit{} & \unit{D} & \unit{D} &\\\cline{1-5}
    relative to Sun & \unit{} & \unit{} & \unit{Sun} & \unit{Sun} &\\\cline{1-5}
    channel & \unit{} & \unit{chan} & \unit{} & \unit{chan} &\\\cline{1-5}
    bin & \unit{} & \unit{bin} & \unit{} & \unit{bin} &\\\cline{1-5}
    voxel & \unit{} & \unit{voxel} & \unit{} & \unit{voxel} &\\\cline{1-5}
    bit & \unit{} & \unit{} & \unit{bit} & \unit{bit} &\\\cline{1-5}
    byte & \unit{} & \unit{byte} & \unit{byte} & \unit{byte} &\\\cline{1-5}
    adu & \unit{} & \unit{} & \unit{} & \unit{adu} &\\\cline{1-5}
    beam & \unit{} & \unit{} & \unit{} & \unit{beam} &\\\hline
     & \unit{} & \unit{Crab} avoid use & \unit{} & \unit{} & not used \\\hline
    No unit, dimensionless & \unit{} & blank string & \unit{-} & \unit{} & empty string \\\hline
    Unitless in percent & \unit{} &  & \unit{\%} & \unit{} & \unit{} \\\hline
    unknown & \unit{} & {\tiny\unit{UNKNOWN}} & \unit{} & \unit{} & \unit{?} \\\hline
%  \end{tabular}
\end{tabular}
  \caption{Comparison of other symbols.}
  \label{tabx:comparUnitOther}
\end{table}
%\end{longtable}

\clearpage

%\subsection{Combination of symbols}

%\begin{longtable}[th]{|p{0.2\linewidth}|p{0.15\linewidth}|p{0.12\linewidth}|p{0.12\linewidth}|p{0.12\linewidth}|p{0.15\linewidth}|}
%\hline
%    & IAU & OGIP  & StdCats & FITS  & VOUnits\\\hline
%    %\multicolumn{6}{|c|}{Compound units} \\\hline
%    Multiplication & space, except if previous unit ends with superscript; dot (\unit{.}) may be used
%    	& one or more spaces OR one asterix (\unit{*}) with optional spaces on either side 
%	& dot (\unit{.}), no space 
%	& single space OR asterix (\unit{*}, no spaces) OR dot (\unit{.}, no spaces)& \\\hline
%    Division & per. Use negative index or solidus (\unit{/}) 
%    	& slash (\unit{/}) with optional spaces on either side, space not recommended after / OR negative index
%	& \unit{/} with no spaces 
%	& \unit{/} with no spaces 
%	& \\\hline\hline
%    Use of multiple / & MUST never two 
%    	& allowed 
%	& allowed 
%	& may be used, discouraged, math precedence rule & \\\hline\hline
%    \unit{sym} raised to the power $y$ & superscript 
%    	& \unit{sym**($y$)} parenthesis optional if $y>0$ 
%	& nothing: \unit{sym$y$} use +/- sign for \unit{10+21} 
%	& \unit{sym$y$} OR \unit{sym**($y$)} OR \unit{sym\^{}($y$)}, no space & \\\hline\hline
%    Exponential of \unit{sym} &  & \unit{exp(sym)} &  & \unit{exp(sym)} & \\\hline\hline
%    Natural log of \unit{sym} &  & \unit{ln(sym)} &  & \unit{ln(sym)} & \\\hline\hline
%    Decimal log of \unit{sym} &  & \unit{log(sym)} & \unit{[sym]} & \unit{log(sym)} dimensionless argument & \\\hline\hline
%    Square root of \unit{sym} &  & \unit{sqrt(sym)} &  & \unit{sqrt(sym)} & \\\hline\hline
%    Other math &  & {\small \unit{sin(sym), cos(sym), tan(sym), asin(sym), acos(sym), atan(sym), sinh(sym), cosh(sym), tanh(sym)} } &  &  & \\\hline\hline
%    ( ) &  & any allowed & allowed & optional around powers & \\\hline\hline
%    powers & superscripts & decimal and integer fractions allowed & integers only & integer (sign and () optional), OR decimal or ratio between () & \\\hline
%  \caption{Comparison of mathematical expressions and symbols combinations.}
%  \label{tabx:comparUnitCombine}
%%\end{table}
%\end{longtable}
\begin{longtable}[th]{|p{0.2\linewidth}|p{0.2\linewidth}|p{0.12\linewidth}|p{0.12\linewidth}|p{0.22\linewidth}|}
\hline
    & IAU & OGIP  & StdCats & FITS \\\hline
    %\multicolumn{6}{|c|}{Compound units} \\\hline
    Multiplication & space, except if previous unit ends with superscript; dot (\unit{.}) may be used\raggedright
    	& one or more spaces OR one asterisk (\unit{*}) with optional spaces on either side\raggedright 
	& dot (\unit{.}), no space 
	& single space OR asterisk (\unit{*}, no spaces) OR dot (\unit{.}, no spaces) \\\hline
    Division & per. Use negative index or solidus (\unit{/})\raggedright
    	& solidus (\unit{/}) with optional spaces on either side, space not recommended after / OR negative index\raggedright
	& \unit{/} with no spaces 
	& \unit{/} with no spaces  \\\hline\hline
    Use of multiple / & MUST never use two /\raggedright 
    	& allowed 
	& allowed 
	& may be used, discouraged, math precedence rule \\\hline\hline
    \unit{sym} raised to the power $y$ & superscript 
    	& \unit{sym**($y$)} parenthesis optional if $y>0$ 
	& nothing: \unit{sym$y$} use +/- sign for \unit{10+21} 
	& \unit{sym$y$} OR \unit{sym**($y$)} OR \unit{sym\^{}($y$)}, no space \\\hline\hline
    Exponential of \unit{sym} &  & \unit{exp(sym)} &  & \unit{exp(sym)} \\\hline\hline
    Natural log of \unit{sym} &  & \unit{ln(sym)} &  & \unit{ln(sym)} \\\hline\hline
    Decimal log of \unit{sym} &  & \unit{log(sym)} & \unit{[sym]} & \unit{log(sym)} dimensionless argument \\\hline\hline
    Square root of \unit{sym} &  & \unit{sqrt(sym)} &  & \unit{sqrt(sym)} \\\hline\hline
    Other math &  & {\small \unit{sin(sym), cos(sym), tan(sym), asin(sym), acos(sym), atan(sym), sinh(sym), cosh(sym), tanh(sym)} } &  & not used \\\hline\hline
    ( ) &  & any allowed & allowed & optional around powers \\\hline\hline
    powers & superscripts & decimal and integer fractions allowed & integers only & integer (sign and () optional), OR decimal or ratio between () \\\hline
    Numeric factor & not used & should be avoided; only powers of 10 allowed; should precede any unit string & allowed & optional 10**k, 10\verb|^|k, or 10$\pm$k \\\hline\hline
  \caption{Comparison of mathematical expressions and symbol combinations.}
  \label{tabx:comparUnitCombine}
%\end{table}
\end{longtable}


\section{Formal grammars\label{appx:grammar}}
% These grammars are extracted from http://bitbucket.org/nxg/unity
% revision f9b7e66b0068

In this section we provide formal (yacc-style) grammars for the four
ASCII-based syntaxes discussed in this document.  The FITS, OGIP and
CDS grammars are not normative: the corresponding specification
documents do not provide grammars, and instead describe the syntaxes
in text, so that the grammars here are deductions from the
specification text.
This unfortunately means that some of these syntaxes are ambiguous.
These ambiguities are discussed in the sections below.  We recommend
that VO applications parse these syntaxes in a way which is consistent
with the grammars here.
%
The grammar for the VOUnits syntax, in \prettyref{appx:vougrammar}, is normative.

We believe that the grammars below are such that if a string 
successfully parses in two distinct grammars, it means the same in
both.

The grammars here are from the `Unity' package at
\url{https://bitbucket.org/nxg/unity}, which includes machine-readable
grammars, lists of recommended units, and a collection of test cases.  These are also extracted in
machine-readable form
at \url{https://code.google.com/p/volute/source/browse/trunk/projects/std-vounits/unity-grammars.zip}.

In these grammars, the terminals are as given in
\prettyref{tabx:terminals}.
\begin{table}[ht]
%\def\={\raggedright\tabularnewline\relax} % avoid problem with \raggedright redefining \\
%\let\=\tabularnewline
\begin{tabular}{rp{9cm}}
\texttt{CARET}&the \texttt{\^{}} character\tabularnewline
\texttt{DIVISION}&the solidus, \texttt{/}\tabularnewline
\texttt{DOT}&the dot/period/full-stop character\tabularnewline
\texttt{FLOAT}&a string matching the regular expression
       \texttt{[-+]?[0-9]+\textbackslash.[0-9]+}\raggedright\tabularnewline
\texttt{LIT10}&a literal string `\texttt{10}' (without quotes).\tabularnewline
\texttt{OPEN\_P} / \texttt{CLOSE\_P}&parentheses\tabularnewline
\texttt{SIGNED\_INTEGER}&an integer with a required leading sign\raggedright\tabularnewline
\texttt{STAR}&the asterisk\tabularnewline
\texttt{STARSTAR}&a pair of asterisks, \texttt{**}\tabularnewline
\texttt{STRING}&a sequence of letters (a--z and A--Z) plus the percent sign\raggedright\tabularnewline
\texttt{UNSIGNED\_INTEGER}&an integer with no leading sign\raggedright\tabularnewline
\texttt{WHITESPACE}&a non-empty string of space characters (no other whitespace)\raggedright\tabularnewline
\end{tabular}
\caption{\label{tabx:terminals}The terminals used in the grammars}
\end{table}

\subsection{FITS}
\label{appx:fitsgrammar}

For the FITS units syntax, see section~4.3 of~\cite{pence10}, and its
associated tables.  Our preferred FITS grammar is in
\prettyref{tabx:fitsgrammar}.

\begin{table}[ht]
\verbatiminput{unity-grammars/unity-fits.txt}
\caption{\label{tabx:fitsgrammar}The FITS grammar}
\end{table}

As noted above in \prettyref{sec:fitsquote}, 
the FITS specification isn't completely clear on the topic of 
solidi, saying `[t]he IAU style manual forbids
the use of more than one solidus (/) character in a units
string. However, since normal mathematical precedence rules apply
in this context, more than one solidus may be used but is
discouraged'.  This does not really resolve the question of whether, for
example, \texttt{kg/m s} should be parsed as \units{kg~m^{-1}~s^{-1}}
or as \units{kg~m^{-1}~s}, since this is a question of both operator
precedence and (left-)associativity, where there might be different
rules internationally, and conflicts between mathematical and
programming-language rules.  Most people would \emph{probably} parse
it as \units{kg~m^{-1}~s^{-1}}, but we trust that most educators would
oblige students to rewrite the expression on the grounds that any
ambiguity is too much.
Here, we resolve the ambiguity by declaring that there can
be only a single expression to the right of the solidus.

It is a consequence of this that nothing can be
successully parsed in two different grammars, with different
meanings.  If the right-hand-side of the division could be a
\texttt{product\_of\_units}, then \texttt{kg /m s} would parse in both
the FITS and OGIP syntaxes,
but mean \units{kg~m^{-1}~s^{-1}} in the FITS syntax, and
\units{kg~m^{-1}~s} in the OGIP one.

Other ambiguities:
\begin{itemize}
\item The FITS specification may or may not be intended to permit 
  \texttt{10+3 /m}, but we don't.
\item It is possible to read the FITS spec as permitting
  \texttt{m\^{}1.5}, without parentheses.  We take it to be
  invalid here.
\end{itemize}

\subsection{OGIP}
\label{appx:ogipgrammar}

For the OGIP units syntax, see \cite{george95}.  Our preferred OGIP
grammar is in \prettyref{tabx:ogipgrammar}.

\begin{table}[ht]
\verbatiminput{unity-grammars/unity-ogip.txt}
\caption{\label{tabx:ogipgrammar}The OGIP grammar}
\end{table}

Specification ambiguities:
\begin{itemize}
\item The OGIP specification permits a space between the leading
  factor and the rest of the unit (by implication from the provided
  examples).
\item OGIP \emph{recommends} having no whitespace after the division
  solidus, but does not forbid it; therefore we permit it in this
  grammar.
\item From its specification text, OGIP appears to permit
  \texttt{str1**y}, where \texttt{y} can be a float, even though none
  of its examples include this.  The same interpretive logic would
  appear to permit \texttt{m**3/2}, but this seems to run too great a
  risk of being misparsed, and we forbid it here.
\item In the same place, the text suggests that \texttt{str1**y} may
  omit the brackets `if~\texttt y is positive', but the context
  suggests that the intention is to permit this if~\texttt y is
  unsigned.  In the grammar here, we permit the omission of the
  brackets only if~\texttt y is unsigned -- that is, \texttt{m**+2},
  like \texttt{m**-2}, is forbidden.
\end{itemize}

\subsection{The CDS grammar}
\label{appx:cdsgrammar}

For the CDS units syntax, see \cite[\S3.2]{cds00}.  Our preferred CDS
grammar is in \prettyref{tabx:cdsgrammar}.

\begin{table}[ht]
\verbatiminput{unity-grammars/unity-cds.txt}
\caption{\label{tabx:cdsgrammar}The CDS grammar}
\end{table}

The \texttt{CDSFLOAT} terminal is a string matching the regular
expression
\begin{quotation}
\texttt{[0-9]+\textbackslash.[0-9]+x10[-+][0-9]+}
\end{quotation}
(that is, something resembling \texttt{1.5x10+11}).

The \texttt{OPEN\_SQ} and \texttt{CLOSE\_SQ} terminals, used to mark
the logarithm of a unit in this syntax, are the open and close square
brackets, `\texttt{[}' and `\texttt{]}'.




\subsection{The VOUnits grammar}
\label{appx:vougrammar}

The VOUnits grammar is defined by this section, the grammar in
\prettyref{tabx:vougrammar} (with the terminals
of \prettyref{tabx:terminals}) and the known units
of \prettyref{tabx:knownunits}.  This grammar is a strict subset of
the FITS and CDS grammars (in the sense that any VOUnit unit string is a
valid FITS and CDS string, too), and it is almost a subset of the OGIP
grammar, except that it uses the dot for multiplication rather than
star.  By design, therefore, a unit specification written to conform
with the VOUnits grammar is immediately readable (with the same
semantics) by a FITS or CDS parser, and almost readable by an OGIP
one.

In particular:
\begin{itemize}
\item The product of units is indicated only by a dot, with no
  whitespace: \texttt{N.m}.
\item Raising a unit to a power is done only with a double-star:
  \texttt{kg.m**2.s**-2}.
\item There may be at most one division sign at the top level of an
  expression.
\end{itemize}

\begin{table}[ht]
\verbatiminput{unity-grammars/unity-vou.txt}
\caption{\label{tabx:vougrammar}The VOUnits grammar}
\end{table}

%\clearpage
\section{Updates of this document}
\begin{itemize}
\item 1.0-20130724: Rephrasing and clarification, responding to RFC
comments.  Update unity grammars to current version (ie, version of 2013-07-22 18:40).
\item 1.0-20130701: Simplified Architecture diagram. Added example
with scientific notation.  Adjusted locations of grammar tables to try
to keep them closer to the associated text.
\item 1.0-20130429: Some restructuring, some rephrasing, and a few layout changes.
\item 1.0-20130225: Large tables from section 3 moved to Appendix A. Short summaries of symbols added
to section 3. Changes to table of known units for consistency with text. Added explanations for units Sun and byte.  
\item 1.0-20121212:
Minor typographical fixes. Added definition of OGIP. Removed last sentence from acknowledgements, which have been moved to the beginning of the document. Changed figure 1 to move Units in Semantics. Added 'discouraged' in first line of \prettyref{tab:VOUnitCombine}. Color change in figure 2 and its label.
\item 1.0-20120801:
Minor typographical fixes
\item 1.0-20120801:
  \begin{itemize}
    \item Included yacc-style grammars in document.
    \end{itemize}
\item 1.0-20120718:
	\begin{itemize}
	\item Removed external tables refs in tables to avoid confusion.
	\item Removed refs to SOFA and NOVAS.
	\item Precision on the "no unit" case in text.
	\item Added formal grammar in annex.
	\item Minor editing and typo fixes.
	\end{itemize}
\item 1.0-20120521:
	\begin{itemize}
	\item Typos fixed, removed F. Bonnarel from authors. 
	\item One sentence rephrased in section 1.2 for clarity.
	\item Clarification of \unit{g} and \unit{kg} issue in \prettyref{sec:baseUnits}.
	\item Added remark on \unit{Pa} in \prettyref{sec:scaleFactors}.
	\item Micro-arcsecond and century explained in \prettyref{tabx:comparUnitAstro}.
	\item \prettyref{tabx:comparUnitDeprecated} completed.
	\item Added numeric factors in \prettyref{tabx:comparUnitCombine} and discussion in text.
	\end{itemize}
\item 1.0-20111216: Major rework of the document.
\item 0.3: initial public release.
%\item version 0.1 to 0.2
% \begin{itemize}
% \item 20090521
%   \begin{itemize}
%    \item added UCD to Quantity in point 4 of subsection ~\ref{sec:labels}
%    \item added `.' in the notation in unit strings in section ~\ref{sec:simpleuse}
%    \item added a sentence on the help of UCd in quantity in section ~\ref{sec:UML}
%   \end{itemize}
%  \item 20090522
%   \begin{itemize}
%    \item clarified the scope of the model in Section \ref{sec:purpose}
%    \item added references in Section \ref{sec:vocab}
%    \item added requirement to be consistent with Quantity DM in
% Section~\ref{sec:quantities}
%    \item minor clarification and subediting
%   \end{itemize}
% \end{itemize}
\end{itemize}


\bibliographystyle{plainnat-eprints}
\bibliography{bib}


\end{document}
